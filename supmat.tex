%contents
\section*{Supplementary Methods}
\label{supplementary_methods}

\subsection*{Sea-level area changes calculations}

Sine wave (\textit{SW}) for sea-level fluctuations:


\[ SW = A sin(\omega(t) + \phi) \]    
\[ \omega = 2 \pi f \]

Surface area ($A$) of a cone:

\[ A = \pi r^2 + \pi r l \]

where $l$ is the length of the hypotenuse of the cone, \textit{h} is the height of the cone, and $r$ is the radius of the cone. 

For the surface area of the upper surface of the cone we drop $\pi r^2$.

\[ A = \pi r l \]

Express the length $l$ in terms of $r$ and replace $l$ in the surface area equation:

\[ l = \frac{r}{cos \theta} \]

\[ A = \frac{\pi r^2}{cos \theta} \]

Calculate the radius $r$ of the cone by rearranging: 

\[ r = \sqrt{\frac{A cos \theta}{\pi}} \]

Calculate the height $h$ from the radius $r$ and the tangent of the angle:

\[ h = tan \theta r \]

Expressing $r$ and $l$ in terms of $h$ we can write area as: 

\[ A = \pi \frac{h}{tan \theta} \frac{h}{sin \theta} \]

\[ A = \pi h^2 \frac{cos \theta}{sin^2 \theta} \]

Given the current height of the cone ($h_0$) and the change in the height given a change in sea-level (\textit{h}) the new surface area of the cone is:

\[ A = \pi (h_0 - h)^2 \frac{cos \theta}{sin^2 \theta} \]

Area of the base of the cone instead to model the projection area (area as measured from an aerial view) commonly used in island biogeography. \\

Base of the cone: 

\[ A = \pi r^2 \]

Re-write with $r$ in terms of $h$:

\[ A = \pi \left( \frac{h}{tan \theta} \right)^2 \]

Given sea-level change ($h$): 

\[ A = \pi \left( \frac{h_0 - h}{tan \theta} \right) ^2 \]

Therefore, the relationship between surface area of the cone and the base of the cone is proportional and differs by a sine term.

\clearpage

\section*{Supplementary Figures}

\begin{figure}[ht]
    \centering
    \includegraphics[width=\textwidth]{runtime_ed95_corr.png}
    \caption{Pearson correlation between $ED_{95}$ statistic and computational run time in the robustness pipeline (see Fig. \ref{fig:pipeline}, main text) for (a) $\Delta$STT, (b) $\Delta$ESTT, (c) $\Delta$NESTT, (d) number of species, (e) number of colonists for all parameter spaces. Pearson’s correlation coefficient for (a) is -0.12, (b) is -0.05, (c) is -0.09, (d) is 0.13, (e) is 0.11.}
    \label{fig:runtime_ed95_corr}
\end{figure}

\begin{figure}
    \centering
    \includegraphics[width=\textwidth]{JBI-21-0508_FigS2.png}
    \caption{Strip charts showing the distributions of the endemic $ED_{95}$ statistic ($\Delta$ESTT $ED_{95}$) for each combination of hyperparameters ($d$ and $x$ controlling the effect of area on the rates of cladogenesis and extinction respectively). One point represents the $ED_{95}$ for a single parameter set with the specified hyperparameters on the \textit{x}-axis. All plots have a dashed line at 0.05 which is the null expectation of the $ED_{95}$. (a) $\Delta$ESTT $ED_{95}$ statistic for oceanic ontogeny. (b) $\Delta$ESTT $ED_{95}$ statistic for oceanic sea-level. (c) $\Delta$ESTT $ED_{95}$ statistic for oceanic ontogeny and sea-level. Sample size for Maui Nui (blue) on each strip is given in the \textit{x}-axis label by N\textsubscript{M}. Sample size for Kaua'i (pink) on each strip is given in the \textit{x}-axis label by N\textsubscript{K}. See Fig. \ref{tab:oceanic_ontogeny_young}-\ref{tab:oceanic_ontogeny_sea_level_old} for parameter combinations.}
    \label{fig:Hyperparameters_endemic}
\end{figure}

\begin{figure}
    \centering
    \includegraphics[width=\textwidth]{JBI-21-0508_FigS3.png}
    \caption{Strip charts showing the distributions of the non-endemic $ED_{95}$ statistic ($\Delta$NESTT $ED_{95}$) for each combination of hyperparameters ($d$ and $x$ controlling the effect of area on the rates of cladogenesis and extinction respectively). One point represents the $ED_{95}$ for a single parameter set with the specified hyperparameters on the \textit{x}-axis. All plots have a dashed line at 0.05 which is the null expectation of the $ED_{95}$. (a) $\Delta$NESTT $ED_{95}$ statistic for oceanic ontogeny. (b) $\Delta$NESTT $ED_{95}$ statistic for oceanic sea-level. (c) $\Delta$NESTT $ED_{95}$ statistic for oceanic ontogeny and sea-level. Sample size for Maui Nui (blue) on each strip is given in the \textit{x}-axis label by N\textsubscript{M}. Sample size for Kaua'i island (pink) on each strip is given in the \textit{x}-axis label by N\textsubscript{K}. See Fig. \ref{tab:oceanic_ontogeny_young}-\ref{tab:oceanic_ontogeny_sea_level_old} for parameter combinations.}
    \label{fig:Hyperparameters_nonendemic}
\end{figure}
 
\begin{figure}
    \centering
    \includegraphics[width=\textwidth]{JBI-21-0508_FigS4.png}
    \caption{Strip charts showing the distribution of the $ED_{95}$ statistics for all clade-specific diversity-dependent scenarios calculated for the $\Delta$STT (a-b), $\Delta$ESTT (c-d), $\Delta$NESTT (e-f), number of species at the present (N Spec) (g-h), and number of colonists at the present (N Col) (i-j) for each geodynamic scenario. The scenarios are: oceanic ontogeny, oceanic sea-level, and oceanic ontogeny sea-level (left), as well as two continental scenarios: continental island and continental land-bridge (right). Each point represents the $ED_{95}$ for a single parameter setting. Dashed line at 0.05 is the null expectation of the $ED_{95}$ error. N shows the sample size for each strip on the \textit{x}-axis. See Fig. \ref{tab:oceanic_ontogeny_young}-\ref{tab:continental_lb_old} for parameter combinations.}
    \label{fig:facet_scenario_cs}
\end{figure}

\begin{figure}
    \centering
    \includegraphics[width=\textwidth]{JBI-21-0508_FigS5.png}
    \caption{Strip charts showing the distribution of the $ED_{95}$ statistics for all island-wide diversity-dependent scenarios calculated for the $\Delta$STT (a-b), $\Delta$ESTT (c-d), $\Delta$NESTT (e-f), number of species at the present (N Spec) (g-h), and number of colonists at the present (N Col) (i-j) for each geodynamic scenario. The scenarios are: oceanic ontogeny, oceanic sea-level, and oceanic ontogeny sea-level (left), as well as two continental scenarios: continental island and continental land-bridge (right). Each point represents the $ED_{95}$ for a single parameter setting. Dashed line at 0.05 is the null expectation of the $ED_{95}$ error. N shows the sample size for each strip on the \textit{x}-axis. See Fig. \ref{tab:oceanic_ontogeny_young}-\ref{tab:continental_lb_old} for parameter combinations.}
    \label{fig:facet_scenario_iw}
\end{figure}

\begin{figure}
    \centering
    \includegraphics[width=\textwidth]{JBI-21-0508_FigS6.png}
    \caption{Strip charts showing the distribution of the $ED_{95}$ statistics for all diversity-independent scenarios calculated for the $\Delta$STT (a-b), $\Delta$ESTT (c-d), $\Delta$NESTT (e-f), number of species at the present (N Spec) (g-h), and number of colonists at the present (N Col) (i-j) for each geodynamic scenario. The scenarios are: oceanic ontogeny, oceanic sea-level, and oceanic ontogeny sea-level (left), as well as two continental scenarios: continental island and continental land-bridge (right). Each point represents the $ED_{95}$ for a single parameter setting. Dashed line at 0.05 is the null expectation of the $ED_{95}$ error. N shows the sample size for each strip on the \textit{x}-axis. See Fig. \ref{tab:oceanic_ontogeny_young}-\ref{tab:continental_lb_old} for parameter combinations.}
    \label{fig:facet_scenario_di}
\end{figure}

\begin{figure}
    \centering
    \includegraphics[width=\textwidth]{JBI-21-0508_FigS7.png}
    \caption{Strip charts showing the distributions of the $ED_{95}$ statistic for each combination of island gradients for oceanic sea-level (without ontogeny). One point represents the $ED_{95}$ for a single parameter set with the specified island gradient on the \textit{x}-axis. All plots have a dashed line at 0.05 which is the null expectation of the $ED_{95}$. (a) $\Delta$STT $ED_{95}$ statistic, (b) $\Delta$ESTT $ED_{95}$ statistic, (c) $\Delta$NESTT $ED_{95}$ statistic for oceanic sea-level. Sample size for Maui Nui (blue) on each strip is given in the \textit{x}-axis label by N\textsubscript{M}. Sample size for Kaua'i (pink) on each strip is given in the \textit{x}-axis label by N\textsubscript{K}. See Fig. \ref{tab:oceanic_sea_level_young}-\ref{tab:oceanic_sea_level_old} for parameter combinations.}
    \label{fig:oceanic_sea_level_gradient_nltt}
\end{figure}

\begin{figure}
    \centering
    \includegraphics[width=\textwidth]{JBI-21-0508_FigS8.png}
    \caption{Strip charts showing the distributions of the $ED_{95}$ statistic for each combination of island gradients for oceanic ontogeny and sea-level. One point represents the $ED_{95}$ for a single parameter set with the specified island gradient on the \textit{x}-axis. All plots have a dashed line at 0.05 which is the null expectation of the $ED_{95}$. (a) $\Delta$STT $ED_{95}$ statistic, (b) $\Delta$ESTT $ED_{95}$ statistic, (c) $\Delta$NESTT $ED_{95}$ statistic for oceanic ontogeny sea-level. Sample size for Maui Nui (blue) on each strip is given in the \textit{x}-axis label by N\textsubscript{M}. Sample size for Kaua'i (pink) on each strip is given in the \textit{x}-axis label by N\textsubscript{K}.  See Fig. \ref{tab:oceanic_ontogeny_sea_level_young}-\ref{tab:oceanic_ontogeny_sea_level_old} for parameter combinations.}
    \label{fig:oceanic_ontogeny_sea_level_gradient_nltt}
\end{figure}

\begin{figure}
    \centering
    \includegraphics[width=\textwidth]{JBI-21-0508_FigS9.png}
    \caption{Strip charts showing the distributions of the $ED_{95}$ statistic for number of species at the present (at the end of the simulation) for each combination of hyperparameters ($d$ and $x$ controlling the effect of area on the rates of cladogenesis and extinction respectively). One point represents the $ED_{95}$ for a single parameter set with the specified hyperparameters on the \textit{x}-axis. All plots have a dashed line at 0.05 which is the null expectation of the $ED_{95}$. (a) $ED_{95}$ statistic for number of species at the present for oceanic ontogeny. (b) $ED_{95}$ statistic for number of species at the present for oceanic sea-level. (c) $ED_{95}$ statistic for number of species at the present for oceanic ontogeny and sea-level. Sample size for Maui Nui (blue) on each strip is given in the \textit{x}-axis label by N\textsubscript{M}. Sample size for Kaua'i (pink) on each strip is given in the \textit{x}-axis label by N\textsubscript{K}. See Fig. \ref{tab:oceanic_ontogeny_young}-\ref{tab:oceanic_ontogeny_sea_level_old} for parameter combinations.}
    \label{fig:Hyperparameters_num_spec}
\end{figure}

\begin{figure}
    \centering
    \includegraphics[width=\textwidth]{JBI-21-0508_FigS10.png}
    \caption{Strip charts showing the distributions of the $ED_{95}$ statistic for number of colonists at the present (at the end of the simulation) for each combination of hyperparameters ($d$ and $x$ controlling the effect of area on the rates of cladogenesis and extinction respectively). One point represents the $ED_{95}$ for a single parameter set with the specified hyperparameters on the \textit{x}-axis. All plots have a dashed line at 0.05 which is the null expectation of the $ED_{95}$. (a) $ED_{95}$ statistic for number of colonists at the present for oceanic ontogeny. (b) $ED_{95}$ statistic for number of colonists at the present for oceanic sea-level. (c) $ED_{95}$ statistic for number of colonists at the present for oceanic ontogeny and sea-level. Sample size for Maui Nui (blue) on each strip is given in the \textit{x}-axis label by N\textsubscript{M}. Sample size for Kaua'i (pink) on each strip is given in the \textit{x}-axis label by N\textsubscript{K}. See Fig. \ref{tab:oceanic_ontogeny_young}-\ref{tab:oceanic_ontogeny_sea_level_old} for parameter combinations.}
    \label{fig:Hyperparameters_num_col}
\end{figure}

\begin{figure}
    \centering
    \includegraphics[width=\textwidth]{JBI-21-0508_FigS11.png}
    \caption{Strip charts showing the distributions of the $ED_{95}$ statistic for the number of species at the present (N Spec) for each combination of island gradients for oceanic sea-level. One point represents the $ED_{95}$ for a single parameter set with the specified island gradient on the \textit{x}-axis. All plots have a dashed line at 0.05 which is the null expectation of the $ED_{95}$. $ED_{95}$ statistic for number of species at the present for oceanic sea-level. Sample size for Maui Nui (blue) on each strip is given in the \textit{x}-axis label by N\textsubscript{M}. Sample size for Kaua'i (pink) on each strip is given in the \textit{x}-axis label by N\textsubscript{K}. See Fig. \ref{tab:oceanic_sea_level_young}-\ref{tab:oceanic_sea_level_old} for parameter combinations.}
    \label{fig:Island_gradient_sea_level_num_spec}
\end{figure}

\begin{figure}
    \centering
    \includegraphics{JBI-21-0508_FigS12.png}
    \caption{Strip charts showing the distributions of the $ED_{95}$ statistic for the number of species at the present (N Spec) for each combination of island gradients for oceanic ontogeny sea-level. One point represents the $ED_{95}$ for a single parameter set with the specified island gradient on the \textit{x}-axis. All plots have a dashed line at 0.05 which is the null expectation of the $ED_{95}$. $ED_{95}$ statistic for number of species at the present for oceanic ontogeny sea-level. Sample size for Maui Nui (blue) on each strip is given in the \textit{x}-axis label by N\textsubscript{M}. Sample size for Kaua'i (pink) on each strip is given in the \textit{x}-axis label by N\textsubscript{K}. See Fig. \ref{tab:oceanic_ontogeny_sea_level_young}-\ref{tab:oceanic_ontogeny_sea_level_old} for parameter combinations.}
    \label{fig:Island_gradient_ontogeny_sea_level_num_spec}
\end{figure}

\begin{figure}
    \centering
    \includegraphics{JBI-21-0508_FigS13.png}
    \caption{Strip charts showing the distributions of the $ED_{95}$ statistic for each combination of continental sampling parameters ($x_s$ and $x_n$ controlling the sampling probability of a species being initially present on the island and the sampling probability of species present initially on the island being non-endemic). One point represents the $ED_{95}$ for a single parameter set with the specified sampling parameters on the \textit{x}-axis. All plots have a dashed line at 0.05 which is the null expectation of the $ED_{95}$. (a) $\Delta$STT $ED_{95}$ statistic, (b) $\Delta$ESTT $ED_{95}$ statistic, (c) $\Delta$NESTT $ED_{95}$ statistic for continental land-bridge. Sample size for young island (green) on each strip is given in the \textit{x}-axis label by N\textsubscript{Y}. Sample size for old island (yellow) on each strip is given in the \textit{x}-axis label by N\textsubscript{O}. See Fig. \ref{tab:continental_lb_young}-\ref{tab:continental_lb_old} for parameter combinations.}
    \label{fig:continental_land_bridge_sample_facet}
\end{figure}

\begin{figure}
    \centering
    \includegraphics{JBI-21-0508_FigS14.png}
    \caption{Strip charts showing the distributions of the $ED_{95}$ statistic for number of species (N Spec) (a) and colonists (N Col) (b) at the present for each combination of continental sampling parameters ($x_s$ and $x_n$ controlling the sampling probability of a species being initially present on the island and the sampling probability of species present initially on the island being non-endemic). One point represents the $ED_{95}$ for a single parameter set with the specified sampling parameters on the \textit{x}-axis. All plots have a dashed line at 0.05 which is the null expectation of the $ED_{95}$. (a) $\Delta$STT $ED_{95}$ statistic, (b) $\Delta$ESTT $ED_{95}$ statistic, (c) $\Delta$NESTT $ED_{95}$ statistic for continental land-bridge. Sample size for young island (green) on each strip is given in the \textit{x}-axis label by N\textsubscript{Y}. Sample size for old island (yellow) on each strip is given in the \textit{x}-axis label by N\textsubscript{O}. See Fig. \ref{tab:continental_lb_young}-\ref{tab:continental_lb_old} for parameter combinations.}
    \label{fig:continental_land_bridge_sample_spec_col_facet_}
\end{figure}

\begin{figure}
    \centering
    \includegraphics{JBI-21-0508_FigS15.png}
    \caption{Strip charts showing the distributions of the $ED_{95}$ statistic for number of species and colonists at the present for each combination of continental sampling parameters ($x_s$ and $x_n$ controlling the sampling probability of a species being initially present on the island and the sampling probability of species present initially on the island being non-endemic). One point represents the $ED_{95}$ for a single parameter set with the specified sampling parameters on the \textit{x}-axis. All plots have a dashed line at 0.05 which is the null expectation of the $ED_{95}$. Sample size for young island (green) on each strip is given in the \textit{x}-axis label by N\textsubscript{Y}. Sample size for old island (yellow) on each strip is given in the \textit{x}-axis label by N\textsubscript{O}. Sample size for the ancient island (beige) on each strip is given in the \textit{x}-axis label by N\textsubscript{A}. See Fig. \ref{tab:continental} for parameter combinations.}
    \label{fig:continental_spec_col_facet_}
\end{figure}

\clearpage

\section*{Supplementary Tables}

\begin{table}[ht]
    \centering
    \caption{Parameter space for oceanic ontogeny for Maui Nui. Parameter space consists of each combination of the model parameters, except carrying capacities, for which only one is chosen for each parameter set. Geological data from \cite{lim_true_2017}.}
    \begin{tabular}{ c | c }
        \multicolumn{2}{c}{\textbf{Oceanic ontogeny Maui Nui}} \\
        \textbf{Model parameters} & \textbf{Parameter value} \\ 
        \hline
        \hline
        Time & 2.55 \\
        \hline
        Mainland species pool & 1000 \\
        \hline
        Cladogenesis & 0.02, 0.04 \\
        \hline
        Extinction & 0.975, 1.95 \\
        \hline
        Carrying capacity (CS) & 0.001, 0.01 \\
        \hline
        Carrying capacity (IW) & 0.01, 0.1 \\
        \hline
        Carrying capacity (DI) & $\infty$ \\
        \hline
        Colonisation & 0.03363, 0.06726 \\
        \hline
        Anagenesis & 0.0295, 0.059 \\
        \hline
        $x$ & 0.075, 0.15 \\
        \hline
        $d$ & 0.1108, 0.2216 \\
        \hline
        Max area & 13500 \\
        \hline
        Current area & 3155 \\
        \hline
        Peak time & 0.53 \\
        \hline
        Total island age & 2.864 \\
    \end{tabular}
    \label{tab:oceanic_ontogeny_young}
\end{table}

\begin{table}[ht]
    \centering
    \caption{Parameter space for oceanic ontogeny for Kaua'i. Parameter space consists of each combination of the model parameters, except carrying capacities, for which only one is chosen for each parameter set.  Geological data from \cite{lim_true_2017}.}
    \begin{tabular}{ c | c }
        \multicolumn{2}{c}{\textbf{Oceanic ontogeny Kaua'i}} \\
        \textbf{Model parameters} & \textbf{Parameter value} \\ 
        \hline
        \hline
        Time & 6.15 \\
        \hline
        Mainland species pool & 1000 \\
        \hline
        Cladogenesis & 0.02, 0.04 \\
        \hline
        Extinction & 0.975, 1.95 \\
        \hline
        Carrying capacity (CS) & 0.001, 0.01 \\
        \hline
        Carrying capacity (IW) & 0.01, 0.1 \\
        \hline
        Carrying capacity (DI) & $\infty$ \\
        \hline
        Colonisation & 0.03363, 0.06726 \\
        \hline
        Anagenesis & 0.0295, 0.059 \\
        \hline
        $x$ & 0.075, 0.15 \\
        \hline
        $d$ & 0.1108, 0.2216 \\
        \hline
        Max area & 3787 \\
        \hline
        Current area & 1431 \\
        \hline
        Peak time & 0.27 \\
        \hline
        Total island age & 8.473 \\
    \end{tabular}
    \label{tab:oceanic_ontogeny_old}
\end{table}

\begin{table}[ht]
    \centering
    \caption{Parameter space for oceanic sea-level for Maui Nui. Parameter space consists of each combination of the model parameters, except carrying capacities, for which only one is chosen for each parameter set.}
    \begin{tabular}{ c | c }
        \multicolumn{2}{c}{\textbf{Oceanic sea-level Maui Nui}} \\
        \textbf{Model parameters} & \textbf{Parameter value} \\ 
        \hline
        \hline
        Time & 2.55 \\
        \hline
        Mainland species pool & 1000 \\
        \hline
        Cladogenesis & 0.02, 0.04 \\
        \hline
        Extinction & 0.975, 1.95 \\
        \hline
        Carrying capacity (CS) & 0.001, 0.01 \\
        \hline
        Carrying capacity (IW) & 0.01, 0.1 \\
        \hline
        Carrying capacity (DI) & $\infty$ \\
        \hline
        Colonisation & 0.03363, 0.06726 \\
        \hline
        Anagenesis & 0.0295, 0.059 \\
        \hline
        $x$ & 0.075, 0.15 \\
        \hline
        $d$ & 0.1108, 0.2216 \\
        \hline
        Current area & 3155 \\
        \hline
        Sea-level amplitude & 60 \\
        \hline
        Sea-level frequency & 25.5 \\
        \hline
        Island gradient angle & 80, 85 \\
    \end{tabular}
    \label{tab:oceanic_sea_level_young}
\end{table}

\begin{table}[ht]
    \centering
    \caption{Parameter space for oceanic sea-level Kaua'i. Parameter space consists of each combination of the model parameters, except carrying capacities, for which only one is chosen for each parameter set.}
    \begin{tabular}{ c | c }
        \multicolumn{2}{c}{\textbf{Oceanic sea-level Kaua'i}} \\
        \textbf{Model parameters} & \textbf{Parameter value} \\ 
        \hline
        \hline
        Time & 6.15 \\
        \hline
        Mainland species pool & 1000 \\
        \hline
        Cladogenesis & 0.02, 0.04 \\
        \hline
        Extinction & 0.975, 1.95 \\
        \hline
        Carrying capacity (CS) & 0.001, 0.01 \\
        \hline
        Carrying capacity (IW) & 0.01, 0.1 \\
        \hline
        Carrying capacity (DI) & $\infty$ \\
        \hline
        Colonisation & 0.03363, 0.06726 \\
        \hline
        Anagenesis & 0.0295, 0.059 \\
        \hline
        $x$ & 0.075, 0.15 \\
        \hline
        $d$ & 0.1108, 0.2216 \\
        \hline
        Current area & 1431 \\
        \hline
        Sea-level amplitude & 60 \\
        \hline
        Sea-level frequency & 61.5 \\
        \hline
        Island gradient angle & 80, 85 \\
    \end{tabular}
    \label{tab:oceanic_sea_level_old}
\end{table}

\begin{table}[ht]
    \centering
    \caption{Parameter space for oceanic ontogeny sea-level for Maui Nui. Parameter space consists of each combination of the model parameters, except carrying capacities, for which only one is chosen for each parameter set. Geological data from \cite{lim_true_2017}.}
    \begin{tabular}{ c | c }
        \multicolumn{2}{c}{\textbf{Oceanic ontogeny sea-level Maui Nui}} \\
        \textbf{Model parameters} & \textbf{Parameter value} \\ 
        \hline
        \hline
        Time & 2.55 \\
        \hline
        Mainland species pool & 1000 \\
        \hline
        Cladogenesis & 0.02, 0.04 \\
        \hline
        Extinction & 0.975, 1.95 \\
        \hline
        Carrying capacity (CS) & 0.001, 0.01 \\
        \hline
        Carrying capacity (IW) & 0.01, 0.1 \\
        \hline
        Carrying capacity (DI) & $\infty$ \\
        \hline
        Colonisation & 0.03363, 0.06726 \\
        \hline
        Anagenesis & 0.0295, 0.059 \\
        \hline
        $x$ & 0.075, 0.15 \\
        \hline
        $d$ & 0.1108, 0.2216 \\
        \hline
        Max area & 13500 \\
        \hline
        Current area & 3155 \\
        \hline
        Peak time & 0.53 \\
        \hline
        Total island age & 2.864 \\
        \hline
        Sea-level amplitude & 60 \\
        \hline
        Sea-level frequency & 25.5 \\
        \hline
        Island gradient angle & 80, 85 \\
    \end{tabular}
    \label{tab:oceanic_ontogeny_sea_level_young}
\end{table}

\begin{table}[ht]
    \centering
    \caption{Parameter space for oceanic ontogeny sea-level for Kaua'i. Parameter space consists of each combination of the model parameters, except carrying capacities, for which only one is chosen for each parameter set. Geological data from \cite{lim_true_2017}.}
    \begin{tabular}{ c | c }
        \multicolumn{2}{c}{\textbf{Oceanic ontogeny sea-level Kaua'i}} \\
        \textbf{Model parameters} & \textbf{Parameter value} \\ 
        \hline
        \hline
        Time & 6.15 \\
        \hline
        Mainland species pool & 1000 \\
        \hline
        Cladogenesis & 0.02, 0.04 \\
        \hline
        Extinction & 0.975, 1.95 \\
        \hline
        Carrying capacity (CS) & 0.001, 0.01 \\
        \hline
        Carrying capacity (IW) & 0.01, 0.1 \\
        \hline
        Carrying capacity (DI) & $\infty$ \\
        \hline
        Colonisation & 0.03363, 0.06726 \\
        \hline
        Anagenesis & 0.0295, 0.059 \\
        \hline
        $x$ & 0.075, 0.15 \\
        \hline
        $d$ & 0.1108, 0.2216 \\
        \hline
        Max area & 3787 \\
        \hline
        Current area & 1431 \\
        \hline
        Peak time & 0.27 \\
        \hline
        Total island age & 8.473 \\
        \hline
        Sea-level amplitude & 60 \\
        \hline
        Sea-level frequency & 61.5 \\
        \hline
        Island gradient angle & 80, 85 \\
    \end{tabular}
    \label{tab:oceanic_ontogeny_sea_level_old}
\end{table}

\begin{table}[ht]
    \centering
    \caption{Parameter space for continental young, old and ancient islands. Parameter space consists of each combination of the model parameters, except carrying capacities, for which only one is chosen for each parameter set.}
    \begin{tabular}{ c | c }
        \multicolumn{2}{c}{\textbf{Continental young, old, and ancient islands}} \\
        \textbf{Model parameters} & \textbf{Parameter value} \\ 
        \hline
        \hline
        Time & 2.55, 6.15, 50.00 \\
        \hline
        Mainland species pool & 1000 \\
        \hline
        Cladogenesis & 0.25, 0.5 \\
        \hline
        Extinction & 0.25, 0.5 \\
        \hline
        Carrying capacity (CS) & 5, 10 \\
        \hline
        Carrying capacity (IW) & 50, 100 \\
        \hline
        Carrying capacity (DI) & $\infty$ \\
        \hline
        Colonisation & 0.01, 0.02 \\
        \hline
        Anagenesis & 0.25, 0.5 \\
        \hline
        $x_s$ & 0.01, 0.05 \\
        \hline
        $x_n$ & 0.1, 0.9 \\
    \end{tabular}
    \label{tab:continental}
\end{table}

\begin{table}[ht]
    \centering
    \caption{Parameter space for continental land-bridge young island. Parameter space consists of each combination of the model parameters, except carrying capacities, for which only one is chosen for each parameter set. Colonisation rate 2 (i.e. when the land-bridge is present) is colonisation rate 1 ($\gamma_1$), multiplied by a colonisation rate multiplier, which in our case is two and ten.}
    \begin{tabular}{ c | c }
        \multicolumn{2}{c}{\textbf{Continental land-bridge young island}} \\
        \textbf{Model parameters} & \textbf{Parameter value} \\ 
        \hline
        \hline
        Time & 2.55 \\
        \hline
        Mainland species pool & 1000 \\
        \hline
        Cladogenesis 1 & 0.5, 1.0 \\
        \hline
        Extinction 1 & 0.5, 1.0 \\
        \hline
        Carrying capacity 1 (CS) & 10 \\
        \hline
        Carrying capacity 1 (IW) & 100 \\
        \hline
        Carrying capacity 1 (DI) & $\infty$ \\
        \hline
        Colonisation 1 & 0.01, 0.05 \\
        \hline
        Anagenesis 1 & 1.0 \\
        \hline
        $x_s$ & 0.01, 0.05 \\
        \hline
        $x_n$ & 0.1, 0.9 \\
        \hline
        Cladogenesis 2 & 0.25, 0.5 \\
        \hline
        Extinction 2 & 0.25, 0.5 \\
        \hline
        Carrying capacity 2 (CS) & 10 \\
        \hline
        Carrying capacity 2 (IW) & 100 \\
        \hline
        Carrying capacity 2 (DI) & $\infty$ \\
        \hline
        Colonisation 2 & $\gamma_1$ * 2, $\gamma_1$ * 10 \\ 
        \hline 
        Anagenesis 2 & 0 \\
        \hline
        Shift times & 1.225, 1.325 \\
    \end{tabular}
    \label{tab:continental_lb_young}
\end{table}

\begin{table}[ht]
    \centering
    \caption{Parameter space for continental land-bridge old island. Parameter space consists of each combination of the model parameters, except carrying capacities, for which only one is chosen for each parameter set. Colonisation rate 2 (i.e. when the land-bridge is present) is colonisation rate 1 ($\gamma_1$), multiplied by a colonisation rate multiplier, which in our case is two and ten.}
    \begin{tabular}{ c | c }
        \multicolumn{2}{c}{\textbf{Continental land-bridge old island}} \\
        \textbf{Model parameters} & \textbf{Parameter value} \\ 
        \hline
        \hline
        Time & 6.15 \\
        \hline
        Mainland species pool & 1000 \\
        \hline
        Cladogenesis 1 & 0.5, 1.0 \\
        \hline
        Extinction 1 & 0.5, 1.0 \\
        \hline
        Carrying capacity 1 (CS) & 10 \\
        \hline
        Carrying capacity 1 (IW) & 100 \\
        \hline
        Carrying capacity 1 (DI) & $\infty$ \\
        \hline
        Colonisation 1 & 0.01, 0.05 \\
        \hline
        Anagenesis 1 & 1.0 \\
        \hline
        $x_s$ & 0.01, 0.05 \\
        \hline
        $x_n$ & 0.1, 0.9 \\
        \hline
        Cladogenesis 2 & 0.25, 0.5 \\
        \hline
        Extinction 2 & 0.25, 0.5 \\
        \hline
        Carrying capacity 2 (CS) & 10 \\
        \hline
        Carrying capacity 2 (IW) & 100 \\
        \hline
        Carrying capacity 2 (DI) & $\infty$ \\
        \hline
        Colonisation 2 & $\gamma_1$ * 2, $\gamma_1$ * 10 \\ 
        \hline 
        Anagenesis 2 & 0 \\
        \hline
        Shift times & 3.025, 3.125 \\
    \end{tabular}
    \label{tab:continental_lb_old}
\end{table}
